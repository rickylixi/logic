%!TEX program = xelatex
%!Tex encoding = UTF-8
\RequirePackage{fix-cm} %优化字体
\documentclass[aspectratio=43,11pt,UTF8,colorlinks,compress,openany]{beamer} %aspectratio=169 %169宽屏 43窄屏[handout]

\input{preamble-beamer}

\begin{document}

%%%%%%%%%%%%%%%%%%%%%%%%%%%%
\title{Consciousness and Self-improving Agents}
\author{
	{\includegraphics[width=0.15\textwidth,angle=0,origin=c]{img/lixi.pdf}}\\[1mm]
	{\includegraphics[width=0.15\textwidth,angle=0,origin=c]{img/wechat.jpg}}\\[1mm]
	\normalsize Department of Philosophy\\
	\normalsize Central South University\\
	\normalsize xieshenlixi@163.com\\
	\normalsize \href{https://github.com/rickylixi/logic}{github}
}
\date{\today}
\maketitle
%%%%%%%%%%%% 节前目录 %%%%%%%%%%
\AtBeginSection[]
{
\frame[shrink]{{Contents}
%\begin{multicols}{2}
\tableofcontents[currentsection,hideallsubsections]
%\end{multicols}
}
\addtocounter{framenumber}{-1} %目录页不计页码
}
\AtBeginSubsection[]
{
\frame[shrink]{{Contents}
%\begin{multicols}{2}
\tableofcontents[sectionstyle=show/shaded,subsectionstyle=show/shaded/hide,subsubsectionstyle=hide]
%\end{multicols}
}
\addtocounter{framenumber}{-1} %目录页不计页码
}





\frame{{}
\begin{figure}
\includegraphics[width=\textwidth]{img/map-on-table.jpeg}\caption{One can imagine a detailed floor plan of a room, sitting on a table in the room; this plan has an image of the table on which there is an image of the plan itself. Now introduce the dynamical aspect: the items on the plan are cut out from paper and can be moved to try a different furniture arrangement; in this way the plan models possible states of the world about which it carries information.}
\end{figure}
}

\frame{{Manin --- \href{https://arxiv.org/abs/1709.03114}{\textcolor{fore}{Cognitive Networks}}}
\begin{tcolorbox}
The brain contains inside a map of itself, and some neural information channels in the central neural system:
\begin{itemize}
	\item carry information about the mind itself, i.e., are \textcolor{red}{reflexive};
	\item are capable of modelling states of the mind different from the current one, i.e., possess a \textcolor{red}{modelling function};
	\item can influence the state of the whole mind and through that, the behavior, i.e., possess \textcolor{red}{controlling function}.
\end{itemize}
The reflection of the brain inside itself must be \textcolor{red}{coarse grained}.
\end{tcolorbox}
}

\frame{{侯士达——“我”是个怪圈}
\begin{itemize}
	\item 有没有意识取决于在哪个层级上对结构进行观察。在整合度最高的层级上看,大脑是有意识的。下降到微观粒子层面,意识就不见了。
	\item 意识体是那些在某个描述层级上表现出某种特定类型的循环回路的结构。当一个系统能把外部世界过滤成不同的范畴、并不断向越来越抽象的层级创造新的范畴时,这种循环回路就会逐渐形成。
	\item 当系统能进行自我表征——对自己讲故事——的时候,这种循环回路就逐渐变成了实体的“我”——一个统一的因果主体。
\end{itemize}
}

\frame{{Liar Paradox vs Quine Paradox}
\begin{itemize}
\item 这句话有2个‘这’字,2个‘句’字,2个‘话’字,2个‘有’字,7个‘2’字,11个‘个’字,11个‘字’字,2个‘7’字,3个‘11’字,2个‘3’字。
\item 我在说谎。
\item 把“\texttt{把中的第一个字放到左引号前面,其余的字放到右引号后面,并保持引号及其中的字不变得到的句子是假的。}”中的第一个字放到左引号前面,其余的字放到右引号后面,并保持引号及其中的字不变得到的句子是假的。
\end{itemize}\vspace{-3ex}
	\begin{figure}[H]
		\includegraphics[width=.7\textwidth,angle=0,origin=c]{img/self-reference-agent}
	\end{figure}
}

\frame{{Diagonalization}
	\begin{itemize}
		\item A function $g\colon X\to Z$ is \emph{representable} by  $f\colon X\times Y\to Z$ iff
		\[\exists y\in Y\forall x\in X\colon g(x)= f(x,y)\]
	\end{itemize}
	\setlength\abovedisplayskip{0pt}
	\setlength\belowdisplayskip{0pt}
	\begin{theorem}[Lawvere's Fixpoint Theorem]
		For sets $X, Y, Z$, functions $\beta\colon X\to Y, f\colon X\times Y\to Z, \alpha\colon Z\to Z$, let $g:=\alpha\circ f\circ(1_X,\beta)$. Assume \textcolor{darkgreen}{$\beta$ is surjective}. 
		\begin{enumerate}
			\item If \textcolor{blue}{$\alpha$ has no fixpoint}, then \textcolor{red}{$g$ is not representable by $f$}.
			\item If \textcolor{blue}{$g$ is representable by $f$}, then \textcolor{red}{$\alpha$ has a fixpoint}.
		\end{enumerate}
	\end{theorem}
	\[\xymatrix{X\times Y
	\ar[rr]^f && Z \ar[dd]^\alpha\\
	\\
	X\ar[uu]^{(1_X,\beta)}\ar[rr]_{g}&&Z}\]
}

\frame{{Kleene's Fixpoint Theorem}
	\setlength\abovedisplayskip{0pt}
	\setlength\belowdisplayskip{0pt}
	\begin{theorem}[Kleene's Fixpoint Theorem]
		Given a recursive function $h$, there is an index $e$ s.t.
		\[\varphi_e=\varphi_{h(e)}\]
	\end{theorem}
	\[\xymatrix{\mathbb{N}\times \mathbb{N}
		\ar[rr]^{f} && \left\{\varphi_n\right\}_{n\in\mathbb{N}} \ar[dd]^{\mathscr{E}_h}
		\\
		\\
		\mathbb{N}\ar[uu]^{\Delta}\ar[rr]_{g}&&\left\{\varphi_n\right\}_{n\in\mathbb{N}}}\]
	where $f\colon(m,n)\mapsto\varphi_{\varphi_n(m)}$, and $\mathscr{E}_h\colon\varphi_n\mapsto\varphi_{h(n)}$.\\
	The function $g\colon m\mapsto\varphi_{h\left(\varphi_m(m)\right)}$ is a recursive sequence of partial recursive functions, and thus is representable by $f$. Explicitly,
	\[g(m)=\varphi_{h(\varphi_m(m))}=\varphi_{s(m)}=\varphi_{\varphi_t(m)}=f(m,t)\]
	\[e:=\varphi_t(t)\]
}

\frame{{von Neumann's Self-reproducing Automata}
	\begin{corollary}[von Neumann's Self-reproducing Automata]
		There is a recursive function $\varphi_e$ s.t. $\forall x\colon \varphi_e(x)= e$.
	\end{corollary}
	\[\fbox{\textcolor{blue}{There is a program that outputs its own length.}}\]
	\[\fbox{\textcolor{blue}{There is a program that outputs its own source code.}}\]
	\[\fbox{\textcolor{darkgreen}{DNA / mutation / evolution}}\]
}

\frame{{Introspective Program}\centering
\fbox{\textcolor{red}{There is a program that is totally introspective.}}
\begin{columns}
\column{.6\textwidth}\vspace{-7ex}
\[\fbox{\Large $\bm{\varphi_e=\varphi_{h(e)}}$}\]
\resizebox{.85\textwidth}{!}{
\begin{minipage}{\textwidth}
\begin{table}
\begin{tabu}{c|c}
\hline
\textbf{Self-simulating Computer} &\textbf{Self-consciousness}\\
\hline
\textcolor{darkgreen}{Host Machine} &\textcolor{red}{Experiencing Self}\\
\hline
\textcolor{darkgreen}{Virtual Machine} &\textcolor{red}{Remembering Self}\\
\hline
Hardware &Body\\
\hline
\end{tabu}
\end{table}
\end{minipage}}
\column{.4\textwidth}
	\begin{figure}[H]\vspace{-1ex}
		\begin{center}
			\includegraphics[width=\textwidth,angle=0,origin=c]{img/yoga}
		\end{center}
	\end{figure}
\end{columns}
}

\frame{{}
	\begin{figure}[H]
		\begin{center}
			\includegraphics[width=.7\textwidth,angle=0,origin=c]{img/snow-chicken}
		\end{center}
	\end{figure}
}

\frame{{}\footnotesize
\begin{longtable}{p{0.32\textwidth}|p{0.6\textwidth}}
\hline
说谎者悖论&\textbf{我在说谎。}\\
\hline
Grelling悖论&‘非自谓的’是自谓的吗?\\
\hline
Russell悖论&“不属于自身的集合的集合”属于自身吗?\\
\hline
Berry悖论&我是少于十八个字不可定义的最小数。\\
\hline
Yablo悖论&我下一句及后面所有的句子都是假的。\\
\hline
G\"odel不动点引理&\textbf{我有性质$\alpha$。}\\
\hline
Tarski算术真不可定义定理&我不真。\\
\hline
G\"odel第一不完全性定理&我不可证。\\
\hline
G\"odel-Rosser不完全性定理&对于任何一个关于我的证明,都有一个更短的关于我的否定的证明。\\
\hline
L\"ob定理&如果我可证,那么$\varphi$。\\
\hline
Curry悖论&如果我是真的,那么圣诞老人存在。\\
\hline
Parikh定理&我没有关于自己的长度短于$n$的证明。\\
\hline
Kleene不动点定理&\textbf{我要进行$h$操作。}\\
\hline
Quine悖论&把“把中的第一个字放到左引号前面,其余的字放到右引号后面,并保持引号及其中的字不变得到的句子是假的。”中的第一个字放到左引号前面,其余的字放到右引号后面,并保持引号及其中的字不变得到的句子是假的。\\
\hline
自测量长度程序&我要输出自己的长度。\\
\hline
自复制程序&我要输出自己。\\
\hline
自反省程序&我要回顾自己走过的每一步。\\
\hline
G\"odel机&\textcolor{red}{我要变成能获取更大效用的自己。}\\
\hline
\end{longtable}
}

\frame{{G\"odel Machine}
	\begin{columns}
		\column{.65\textwidth}\vspace{1cm}
			\begin{figure}[!htb]
\resizebox{.7\textwidth}{!}{\begin{minipage}{\textwidth}
				\begin{tikzpicture}[scale=0.5]
					\draw(0,0)circle(6 and 4)node(C){\phantom{\Huge abc$\dfrac{\dfrac{a}{b}}{\dfrac{a}{b}}$}};
					\draw(3,0)circle(3 and 2)node(F){\huge\textcolor{blue}{\quad\textbf{solver}}};
					\draw(16,0)circle(7 and 5)node(P){\Huge\textbf{\underline{Environment}}};
					\begin{scope}
					\clip(3,0)circle(3 and 2);
					\filldraw[lightgreen,fill=lightgreen,nearly transparent](-5,-3)rectangle(6,6);
					\end{scope}
					\begin{scope}
					\clip(0,0)circle(6 and 4);
					\filldraw[gray,fill=gray,nearly transparent](-6,-6)rectangle(12,12);
					\end{scope}
					\node at (0,3){\huge\textcolor{blue}{\textbf{searcher}}};
					\node at (-0.5,0){\Huge\textcolor{purple}{\textbf{\underline{Agent}}}};
					\path (C) edge [very thick,loop above] node {\textcolor{blue}{\huge read/write}} (C);
					\path (P) edge [->,very thick,bend left] node {\huge\textcolor{darkgreen}{ perception}} (F);
					\path (F) edge [->,very thick,bend left] node {\huge\textcolor{darkgreen}{ action}} (P);
				\end{tikzpicture}
\end{minipage}}
\end{figure}
		\column{.45\textwidth}\vspace{-4.5cm}
			\begin{itemize}
				\item \textcolor{blue}{GRL}
				\item \textcolor{blue}{Universal Search}
				\item \textcolor{blue}{Self-Improvement}
				\item \textcolor{blue}{Logic}
			\end{itemize}
	\end{columns}
\textcolor{red}{Disadvantage:} A G\"odel Machine with a badly chosen utility function is motivated to converge to a ``poor'' program. \textcolor{red}{(orthogonality!)}
}

\frame{{G\"odel Machine vs Self-Consciousness vs Free Will?}\vspace{-1ex}
\begin{table}
\begin{tabu}{c|c|c}
\hline
\textbf{Self-simulating Computer} &\textbf{G\"odel Machine} &\textbf{Self-consciousness}\\
\hline
\textcolor{darkgreen}{Host Machine} &\textcolor{blue}{Solver} &\textcolor{red}{Experiencing Self}\\
\hline
\textcolor{darkgreen}{Virtual Machine} &\textcolor{blue}{Searcher} &\textcolor{red}{Remembering Self}\\
\hline
Hardware &Hardware &Body\\
\hline
\end{tabu}
\end{table}\vspace{3ex}
\[\qquad\fbox{\Large $\bm{\varphi_e=\varphi_{h(e)}}$}\]\vspace{-12ex}
\begin{figure}[!htb]\hspace{-0.57\textwidth}
\resizebox{.64\textwidth}{!}{\begin{minipage}{1.1\textwidth}
				\begin{tikzpicture}[scale=0.5]
					\draw(0,0)circle(6 and 3)node(C){\phantom{\Huge abc$\dfrac{\dfrac{a}{b}}{\dfrac{a}{b}}$}};
					\draw(-18,0)circle(1 and 1)node(F){\includegraphics[width=0.57\textwidth,angle=0,origin=c]{img/yoga1}};
					\draw(15,0)circle(5 and 3)node(P){\Large\textbf{\textcolor{red}{Remembering Self}}};
					\node at (0,0){\huge\textcolor{red}{\textbf{Experiencing Self}}};
					\node at (0,2){\Large\textcolor{blue}{\textbf{Solver}}};
					\node at (15,2){\Large\textcolor{blue}{\textbf{Searcher}}};
					\path (P) edge [->,very thick,bend left] node {\huge\textcolor{darkgreen}{}} (C);
					\path (C) edge [->,very thick,bend left] node {\huge\textcolor{darkgreen}{}} (P);
					\path (F) edge [->,very thick,bend left] node {\huge\textcolor{darkgreen}{}} (C);
					\path (C) edge [->,very thick,bend left] node {\huge\textcolor{darkgreen}{}} (F);
				\end{tikzpicture}
\end{minipage}}
\end{figure}\vspace{-1.9cm}
	\[\hspace{1.5cm}\fbox{\textcolor{blue}{self-reference} \textcolor{red}{$\stackrel{?}{\implies}$} \textcolor{blue}{self-improvement}}\]
}

\frame{{G\"odel Machines}
\begin{enumerate}
	\item \emph{one-time} self-improvement: Kleene's fixpoint theorem
	\[\textcolor{red}{\varphi_e= \varphi_{h(e)}}\]
	\item \emph{continuous} self-improvement: Kleene's fixpoint theorem \textcolor{red}{with parameters}
	\[\textcolor{red}{\varphi_{e(y)}=\varphi_{h(e(y),y)}}\]
	\item \emph{uncomputable} case: Kleene's \textcolor{red}{relativized} fixpoint theorem
	\[\textcolor{red}{\varphi_{e(y)}^A=\varphi_{h(e(y),y)}^A}\]
\end{enumerate}
}

\frame{{Limitation}
\begin{enumerate}
	\item G\"odel's first incompleteness theorem / Rice's theorem
	\item G\"odel's second incompleteness theorem
	\[\mathbb{T}\vdash\Box_{\mathbb{T}'}\varphi\to\varphi\implies \mathbb{T}\vdash\mathit{Con}(\mathbb{T}')\]
	\item Legg's incompleteness theorem. \emph{General prediction algorithms must be complex. Beyond a certain complexity they can't be mathematically discovered.}
	\item Complexity: higher-level abstractions --- coarse grained.\\
	\fbox{Learning is to forget!}
\end{enumerate}
}

\frame{{}
\begin{figure}[H]
\begin{center}
\begin{overpic}[scale=0.55]{img/wheeleru.png}
	\put(32,42){\Huge\color{cyan}{\textbf{Thanks}}}
\end{overpic}
\end{center}
\end{figure}
}




%\frame[allowframebreaks]{{References}\printbibliography[heading = bibintoc]}
\end{document}